\documentclass{article}

% Packages
\usepackage{graphicx} 
\usepackage{amsmath}

% Title, author, date
\title{Brighton and Hove Albion FC}
\author{Potop Horia-Ioan}
\date{\today}

\begin{document}

\maketitle

% Table of Contents
\tableofcontents
\newpage
% Introduction
\section{Introducere}
\hspace{0.5cm}Brighton and Hove Albion Football Club este un club de fotbal profesionist englez cu sediul în orașul Brighton and Hove. Ei concurează în Premier League, nivelul superior al sistemului ligii engleze de fotbal. 

% Sections
\section{Istoria clubului}
\hspace{0.5cm}Fondată în 1901, Brighton a jucat fotbalul profesionist timpuriu în Liga de Sud, înainte de a fi aleși în Liga de Fotbal în 1920.  Înainte de perioada actuală, continuă în Premier League, clubul s-a bucurat cea mai mare proeminență între 1979 și 1983 când au jucat în Prima Divizie și au ajuns în finala Cupei FA din 1983, pierzând în fața lui Manchester United după o reluare. Au retrogradat din Prima Divizie în același sezon.

\subsection{Anii 90'}
\hspace{0.5cm}Până la sfârșitul anilor 1990, Brighton era în al patrulea nivel al fotbalului englez și avea probleme financiare. După ce a evitat cu puțin timp retrogradarea din Liga de Fotbal la Conferință în 1997, o preluare a sălii de consiliu a salvat clubul de la lichidare. Promovările 

\subsection{Perioada recenta}
\hspace{0.5cm}Promovările succesive din 2001 și 2002 l-au adus pe Brighton înapoi la nivelul doi, iar în 2011, clubul s-a mutat pe stadionul Falmer după 14 ani fără un teren permanent. În sezonul 2016–17, Brighton a terminat pe locul al doilea în campionatul EFL și a fost astfel promovat în Premier League, punând capăt unei absențe de 34 de ani din clasamentul de vârf. \par În sezonul 2022-2023, Brighton a terminat pe locul șase în Premier League, cel mai mare clasament de top de până acum și s-a calificat în UEFA Europa League; prima lor participare la fotbalul european al cluburilor.
% Equations
\section{Ecuatii}
\subsection{Ecuatii numerotate}
\begin{equation}
    locuri =  30750
\end{equation}
\begin{equation}
    jucatori = 27
\end{equation}

\subsection{Ecuatii nenumerotate in text}
\hspace{0.5cm}În 2020, clubul a prezentat planuri de extindere a stadionului de la \( x=30.750 \) de locuri la \( y = 32.500 \) adică o creștere de \(procent =  \frac{32500 - 30750}{30750} \), inclusiv ospitalitate suplimentară. În 2021, stadionul a fost extins la \( z = 31800 \), cu lucrări suplimentare încă de făcut.


\subsection{Ecuatii nenumerotate pe alt rand}
\hspace{0.5cm}
\[
   Vechime = 122 ani
\]

\[
   Capacitate = 31,876 locuri
\]


% Lists
\section{Liste}
\subsection{Listă numerotată}
\hspace{0.5cm}Lotul echipei la momentul actual este:
\begin{enumerate}
    	\item Bart Verbruggen
    	\item Tariq Lamptey
    	\item Igor Julio
	\item Adam Webster
   	\item Lewis Dunk (captain)
   	\item James Milner
	\item Solly March
    	\item João Pedro
	\item Julio Enciso
   	\item Billy Gilmour
   	\item Pascal Groß (vice-captain)
	\item Adam Lallana
	\item Jakub Moder
    	\item Danny Welbeck
    	\item Valentín Barco
	\item Carlos Baleba
   	\item Kaoru Mitoma
   	\item Jason Steele
	\item Simon Adingra
    	\item Evan Ferguson
	\item Jan Paul van Hecke
   	\item Pervis Estupiñán
   	\item Ansu Fati
	\item Joël Veltman
	\item Tom McGill
	\item Facundo Buonanotte
	\item Jack Hinshelwood

\end{enumerate}

\subsection{Listă nenumerotată}
\hspace{0.5cm}Jucatori imprumutati:
\begin{itemize}
    \item Mahmoud Dahoud
    \item Kjell Scherpen
    \item Adrian Mazilu
\end{itemize}

\subsection{Listă etichetată}
\hspace{0.5cm}Ultimii 3 antrenori
\begin{description}
    \item[Roberto De Zerbi] Antrenor Italian si actualul antrenor al celor de la Brighton
    \item[Graham Potter] Antrenor englez ce a activat si la Chelsea
    \item[Chris Hughton] Antrenor irlandez, a organizat echipa timp de 5 ani
\end{description}



% Table
\section{Tabel}
\begin{table}[h]
    \centering
    \begin{tabular}{|c|c|}
        \hline
        Position & Name \\
        \hline
        Head coach & Roberto De Zerbi \\
	Assistant head coach & Andrea Maldera\\
	First-team coach & Andrew Crofts\\
	Goalkeeping coach & Ricard Segarra\\
	\hline
    \end{tabular}
    \caption{Pozitiile antrenorilor}
    \label{tab:tabel1}
\end{table}


\begin{figure}
	\centering
    \includegraphics[width=0.5\textwidth]{"C:/Users/Horia/Downloads/numefisier.png"}
    \caption{Stema clubului sportiv}
    \label{fig:figura1}
\end{figure}


\section{Determinant}
\[
\begin{vmatrix}
a & b & c & d \\
e & f & g & h \\
i & j & k & l \\
m & n & o & p \\
\end{vmatrix}
\]

\section{Radical din fractie}
\[
\sqrt{\frac{{\sin(x) + \cos(y)}}{{\log(z) + e^t}}}
\]


% Bibliography
\begin{thebibliography}{9}
    \bibitem{ref1} Wikipedia, \emph{Brigthon FC}.
    \bibitem{ref2} transfermarkt, \emph{Brighton and Hove Albion}.
\end{thebibliography}

\end{document}