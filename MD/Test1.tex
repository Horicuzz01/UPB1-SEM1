\documentclass{article}
\usepackage{amsmath}
\usepackage{amssymb}
\usepackage{hyperref}

\title{Informatii curs obligatoriu}
\author{Profesor Titular}
\date{current date}

\renewcommand{\contentsname}{Cuprins}
\renewcommand{\refname}{Bibliografie}

\begin{document}
\maketitle
\tableofcontents

\subsection{Introducere}
Disciplina \textbf{"Generarea si managementul documentelor"} \cite{ref1} reprezinta un curs obligatoriu pentru studentii de la Facultatea de Informatica.

\subsection{Continut teoretic}
Tematicile ..
\begin{itemize}
    \item Generarea documentelor
    \item Tehnologii
    \begin{description}
        \item[[Tehnologia 1]] Latex
        \item[[Tehnologia 2]] HTML
        \item[[Tehnologia 3]] XML
    \end{description}
    \item Gestionarea Documentelor

\end{itemize}

Rezolvati ecuatiile \cite{ref2}:  \(\sum\limits_{i=1}^{n}x = \alpha\) 
\[\int\limits_{-\infty}^{\infty}\frac{x}{2}dx = \sqrt{5}\]

\begin{equation}
    X^2 + X^3 = f(x)
\end{equation}

\begin{pmatrix}
    1 & 2 & 3 \\
    4 & 5 & 6 \\
    7 & 8 & 9

\end{pmatrix}

\arraycolsep=1.4pt\def\arraystretch{1.5}

\begin{array}{ccc}
    1 & 2 & 3 \\
    4 & 5 & 6 \\
    7 & 8 & 9

\end{array}

\begin{tabular}{|c|c|c|}
    \hline
    1 & 2 & 3 \\
    \hline
    4 & 5 & 6 \\
    \hline
    7 & 8 & 9 \\
    \hline
\end{tabular}

\subsection{Concluzie}
Disciplina etc...

\begin{thebibliography}{9}
    \bibitem{ref1} Google \url{https://www.google.com}
    \bibitem{ref2} Wikipedia \url{https://www.wikipedia.org}
\end{thebibliography}
\end{document}
```

    