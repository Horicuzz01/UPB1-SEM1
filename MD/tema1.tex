\documentclass{article}

% Packages
\usepackage[utf8]{inputenc}
\usepackage[romanian]{babel}
\usepackage{graphicx}
\usepackage{amsmath}
\usepackage{enumitem}

% Title
\title{Titlul lucrării}
\author{Autorul}
\date{\today}

\begin{document}

\maketitle

% Header
\section*{Header}
Numele studentului: Numele tău

\section{Secțiunea 1}
Textul secțiunii 1.

\subsection{Subsecțiunea 1.1}
Textul subsecțiunii 1.1.

% Equations
\section{Ecuatii}
\subsection{Ecuatii numerotate}
\begin{equation}
    E = mc^2
\end{equation}

\subsection{Ecuatii nenumerotate in text}
Lorem ipsum dolor sit amet, consectetur adipiscing elit. $E = mc^2$.

\subsection{Ecuatii nenumerotate pe alt rand}
Lorem ipsum dolor sit amet, consectetur adipiscing elit.
\[
    E = mc^2
\]

% Lists
\section{Liste}
\subsection{Listă numerotată}
\begin{enumerate}
    \item Element 1
    \item Element 2
    \item Element 3
\end{enumerate}

\subsection{Listă nenumerotată}
\begin{itemize}
    \item Element 1
    \item Element 2
    \item Element 3
\end{itemize}

\subsection{Listă etichetată}
\begin{description}
    \item[Element 1] Descriere element 1
    \item[Element 2] Descriere element 2
    \item[Element 3] Descriere element 3
\end{description}

% Table
\section{Tabel}
\begin{table}[ht]
    \centering
    \begin{tabular}{|c|c|}
        \hline
        Coloana 1 & Coloana 2 \\
        \hline
        Valoare 1 & Valoare 2 \\
        \hline
    \end{tabular}
    \caption{Descrierea tabelului}
    \label{tab:tabel}
\end{table}

% Bibliography
\begin{thebibliography}{9}
    \bibitem{item1} Autor, \emph{Titlu lucrare}, Anul publicării.
    \bibitem{item2} Autor, \emph{Titlu lucrare}, Anul publicării.
\end{thebibliography}

\end{document}